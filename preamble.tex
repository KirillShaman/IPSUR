%%%%%%%%%%%%%%%%%%%%%%%%%%%%%% User specified LaTeX commands.
% Meta information - fill between {} and do not remove %
% \VignetteIndexEntry{An Introduction to Probability and Statistics Using R}
% \VignetteDepends{}
% \VignetteKeywords{}
% \VignettePackage{IPSUR}
% Package

\usepackage{lmodern}
\renewcommand{\sfdefault}{lmss}
\renewcommand{\ttdefault}{lmtt}

% needed packages
\usepackage{amsmath}
\usepackage{amssymb}
\usepackage{amsthm}
\usepackage{babel}
\usepackage{epsfig}
\usepackage{fixltx2e}
\usepackage{float}
%\usepackage{floatflt}
\usepackage[T1]{fontenc}
\usepackage{footnote}
\usepackage{graphics}
\usepackage{graphicx}
\usepackage[utf8]{inputenc}
\usepackage{latexsym}
\usepackage{longtable}
\usepackage{makeidx}
\usepackage{marvosym}
\usepackage{multicol}
%\usepackage{pslatex}
\usepackage{rotating}
%\usepackage{showidx}
\usepackage{soul}
\usepackage{srcltx}
\usepackage{stmaryrd}
\usepackage{subfig}
\usepackage{textcomp}
%\usepackage{theorem}
\usepackage[subfigure]{tocloft}
\usepackage{txfonts}
\usepackage{upgreek}
\usepackage{url}
\usepackage{varioref}
\usepackage{verbatim}
\usepackage{wasysym}
\usepackage{wrapfig}


% Page setup
\usepackage[paperwidth=7.44in,paperheight=9.69in]{geometry}
\geometry{verbose,tmargin=1in,bmargin=1in,lmargin=1in,rmargin=1in}
\pagestyle{headings}
\setcounter{secnumdepth}{2}
\setcounter{tocdepth}{1}

\makeindex

% PDF settings
\usepackage[hyperref,x11names]{xcolor}
\usepackage[	unicode=true, 
		bookmarks=true, 
		bookmarksnumbered=true, 
		bookmarksopen=true, 
		bookmarksopenlevel=0, 
		breaklinks=true,
		pdfborder={0 0 0},
		backref=page,
		colorlinks=true]{hyperref}
\hypersetup{pdftitle={Introduction to Probability and Statistics Using R},
 		pdfauthor={G. Jay Kerns}, 
		linkcolor=Firebrick4, 
		citecolor=black, 
		urlcolor=SteelBlue4}

% Listings setup
\usepackage{color}
\usepackage{listings}
\lstset{basicstyle={\ttfamily},
	language=R,
	breaklines=true,
	breakatwhitespace=true,
	keywordstyle={\ttfamily},
	numberstyle = {\ttfamily},
	morestring=[b]"
}




%%%%%%%%%%%%%%%%%%%%%%%%%%%%%% LyX specific LaTeX commands.
\providecommand{\LyX}{L\kern-.1667em\lower.25em\hbox{Y}\kern-.125emX\@}
\newcommand{\noun}[1]{\textsc{#1}}
%% Because html converters don't know tabularnewline
\providecommand{\tabularnewline}{\\}

% special logos
\providecommand{\IPSUR}
{\textsc{I\kern 0ex\lower-.3ex\hbox{\small P}\kern -.5ex\lower.4ex\hbox{\footnotesize S}\kern -.25exU}\kern -.1ex\lower .15ex\hbox{\textsf{\large R}}\@}

\providecommand{\IPSURtitle}
{\fontsize{30}{35}\selectfont \textsc{I\kern -.16ex\lower-.5ex\hbox{\Large P}\kern -.5ex\lower.4ex\hbox{\Large S}\kern -.25exU}\kern -.1ex\lower .15ex\hbox{\textsf{\Huge R}}\@}


%  user defined commands
% special operators
\renewcommand{\P}{\mathrm{I\hspace{-1.5pt}P}}
\newcommand{\E}{\mathrm{I\hspace{-1.5pt}E}}
\renewcommand{\vec}[1]{\mbox{\boldmath$#1$}}

% special symbols
\newcommand{\me}{\mathrm{e}}
\newcommand{\R}{\mathbb{R}}
\newcommand{\diff}{\mathrm{d}}
\newcommand{\ybar}{\overline{y}}
\newcommand{\xbar}{\overline{x}}
\newcommand{\Xbar}{\overline{X}}
\newcommand{\Ybar}{\overline{Y}}


\makeatletter


%%%%%%%%%%%%%%%%%%%%%%%%%%%%%% Textclass specific LaTeX commands.
\newcommand{\Rcode}[1]{{\texttt{#1}}}
\newcommand{\Robject}[1]{{\texttt{#1}}}
\newcommand{\Rcommand}[1]{{\texttt{#1}}}
\newcommand{\Rfunction}[1]{{\texttt{#1}}}
\newcommand{\Rfunarg}[1]{{\textit{#1}}}
\newcommand{\Rpackage}[1]{{\textit{#1}}}
\newcommand{\Rmethod}[1]{{\textit{#1}}}
\newcommand{\Rclass}[1]{{\textit{#1}}}
\numberwithin{equation}{section}
\numberwithin{figure}{section}
\theoremstyle{plain}
\ifx\thechapter\undefined
\newtheorem{thm}{Theorem}
\else
\newtheorem{thm}{Theorem}[chapter]
\fi
 \theoremstyle{definition}
  \newtheorem{example}[thm]{Example}
  \theoremstyle{plain}
  \newtheorem{fact}[thm]{Fact}
\newenvironment{lyxcode}
{\par\begin{list}{}{
\setlength{\rightmargin}{\leftmargin}
\setlength{\listparindent}{0pt}% needed for AMS classes
\raggedright
\setlength{\itemsep}{0pt}
\setlength{\parsep}{0pt}
\normalfont\ttfamily}%
 \item[]}
{\end{list}}
  \theoremstyle{definition}
  \newtheorem{xca}[thm]{Exercise}
  \theoremstyle{remark}
  \newtheorem{note}[thm]{Note}
  \theoremstyle{plain}
  \newtheorem{ax}[thm]{Axiom}
  \theoremstyle{plain}
  \newtheorem{prop}[thm]{Proposition}
  \theoremstyle{definition}
  \newtheorem{defn}[thm]{Definition}
  \theoremstyle{remark}
  \newtheorem{rem}[thm]{Remark}
  \theoremstyle{plain}
  \newtheorem{cor}[thm]{Corollary}
  \theoremstyle{plain}
  \newtheorem{assumption}[thm]{Assumption}
  \theoremstyle{remark}
  \newtheorem*{note*}{Note}

\setlength{\cftfignumwidth}{1.5cm}

\@ifundefined{showcaptionsetup}{}{%
 \PassOptionsToPackage{caption=false}{subfig}}
\usepackage{subfig}
\AtBeginDocument{
  \def\labelitemii{\(\circ\)}
}

\makeatother




